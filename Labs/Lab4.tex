\documentclass[12pt]{amsbook}

\usepackage{amssymb} 
\usepackage{graphicx} 
\usepackage{minted}
\usepackage{url}
\usepackage{xcolor}
\definecolor{bookColor}{RGB}{19,9,46}  % 0.90\% of black
\color{bookColor}

\usepackage{lipsum}
\newenvironment{bottompar}{\par\vspace*{\fill}}{\clearpage}


% tekst pisany jest pochyłą czcionkś:
\newtheorem{twierdzenie}{Twierdzenie}[chapter] % to jest główne środowisko, ono musi być zdefiniowane jako pierwsze
\newtheorem{lemat}[twierdzenie]{Lemat} 
% tekst pisany jest prostą czcionką:
\theoremstyle{definition}
\newtheorem{definicja}[twierdzenie]{Definicja}
\newtheorem{obserwacja}[twierdzenie]{Obserwacja}
\newtheorem{stw}[twierdzenie]{Stwierdzenie}
\newtheorem{uwaga}[twierdzenie]{Uwaga}
\newtheorem{przyklad}[twierdzenie]{Przyk\l{}ad}

\usepackage{mathrsfs}
\usepackage[T1]{fontenc}
\usepackage[polish]{babel}
\usepackage[utf8]{inputenc}
\usepackage{lmodern}
\selectlanguage{polish}
\usepackage{listings}
\usepackage{matlab-prettifier}
\lstset{style=Matlab-editor}

\usepackage{scalerel}[2016/12/29]

\newcommand\Searrow{\scaleobj{0.7}{\searrow}}

\setlength\parindent{20pt}

\renewcommand\citeform[1]{{#1}}


\usepackage{geometry}
\geometry{margin=1.25in}
\usepackage{fancyhdr}

\usepackage{xcolor}
\newcommand\textlcsc[1]{\textsc{\MakeLowercase{#1}}}


\begin{document}

\begin{center}
\large{Laboratorium 4}\\
\end{center}



\begin{center}
\textsc{Obliczenia symboliczne}\\
\end{center}

Będziemy korzystać z \textit{\textbf{Symbolic Math Toolbox}}.\\

\begin{center}
Układy równań liniowych\\
\end{center}

Równanie liniowe ma postać:

$$a_1x_1 + a_2x_2 + a_3x_3 + \cdots + a_nx_n = b\\$$

lub ogólniej,

$$Ax=b.\\$$


\begin{center}
Macierze\\
\end{center}

\begin{enumerate}
\item \textbf{reshape}\\

Proszę sprawdzić, co robią polecenia.

\begin{lstlisting}
>> mymat = reshape(1:16,4,4)

>> mymat_transp = reshape(1:16,4,4)'

>> mymat_niekwadratowa = reshape(1:16,2,8)
\end{lstlisting}

\item \textbf{diag} -- przekątna\\

\begin{lstlisting}
>> diag(mymat)
\end{lstlisting}

\item \textbf{diag} można też użyć do stworzenia macierzy diagonalnej o danej przekątnej


\begin{lstlisting}
>> v = 1:4;
>> diag(v)
\end{lstlisting}

\item \textbf{trace} -- ślad macierzy

\begin{lstlisting}
>> trace(mymat)
\end{lstlisting}

\item \textbf{eye} -- czyli macierz identycznościowa

\begin{lstlisting}
>> eye(5)
\end{lstlisting}

\item \textbf{triu, tril} -- macierz górnotrójkątna i dolnotrójkątna.

\begin{lstlisting}
>> triu(mymat)

>> tril(mymat)
\end{lstlisting}

\item \textbf{inv} -- odwrotność macierzy

\begin{lstlisting}
>> a = [1 2; 2 2]

>> ainv = inv(a)

>> a*ainv
\end{lstlisting}

\item \textbf{det} -- wyznacznik macierzy

\begin{lstlisting}
>> det(a)
\end{lstlisting}


\end{enumerate}

Dla układu równań $$Ax = b$$ rozwiązaniem jest $$x = A^{-1}b.$$

\begin{lstlisting}
>> A = [4 -2 1; 1 1 5; -2 3 -1];
>> b = [7;10;2];
>> x = inv(A)*b
\end{lstlisting}


\begin{center}
\textsc{Symboliczne zmienne i wyrażenia}\\
\end{center}

Zmienne:

\begin{lstlisting}
>> a = sym('a');
>> a + a
\end{lstlisting}

Wyrażenia:

\begin{lstlisting}
>> b = sym('x^2');
>> c = sym('x^4');
\end{lstlisting}


Na takich symbolicznych wyrażeniach można wykonywać wszytskie operacje matematyczne.
Proszę wypróbować poniższe przykłady:


\begin{lstlisting}
>> c/b

>> b^3

>> c*b

>> b + sym('4*x^2')

\end{lstlisting}

Ale przy definiowaniu wyrażenia, nie upraszcza się ono automatycznie, tzn.:

\begin{lstlisting}
>> sym('z^3 + 2*z^3')
ans =
z^3 + 2*z^3
\end{lstlisting}

Ale, jeśli $z$ było zdefiniowane jako zmienna symboliczna, to nie potrzeba cudzysłowu wokół wyrażenia i upraszcza się ono od razu:

\begin{lstlisting}
>> z = sym('z');
>> z^3 + 2*z^3
\end{lstlisting}

Jeśli chcemy od razu zdefiniować kilka zmiennych symbolicznych, używamy \textbf{syms}

\begin{lstlisting}
>> syms x y z
\end{lstlisting}

To to samo, co:
\begin{lstlisting}
>> x = sym('x');
>> y = sym('y');
>> z = sym('z');
\end{lstlisting}


\begin{center}
Wielomiany\\
\end{center}

Wbudowane funkcje \textbf{sym2poly} oraz \textbf{poly2sym} służą do konwersji z wyrażeń symbolicznych na wektory i vice versa.

Na przykład:

\begin{lstlisting}
>> myp = [1 2 -4 3];
>> poly2sym(myp)
ans =
x^3+2*x^2-4*x+3
>> mypoly = [2 0 -1 0 5];
>> poly2sym(mypoly)
ans =
2*x^4-x^2+5
>> sym2poly(ans)
ans =
2	0 	-1	 0	 5
\end{lstlisting}

\begin{center}
Upraszczanie wyrażeń\\
\end{center}

Jest bardzo fajne!
Proszę się przekonać:

\begin{lstlisting}
>> x = sym('x');
>> myexpr = cos(x)^2 + sin(x)^2

>> simplify(myexpr)
\end{lstlisting}

Funkcje \textbf{collect}, \textbf{expand}, \textbf{factor}.


\begin{lstlisting}
>> x = sym('x');
>> collect(x^2 + 4*x^3 + 3*x^2)
\end{lstlisting}

\begin{lstlisting}
>> expand((x+2)*(x-1))
ans =
x^2+x-2
>> factor(x^3 + 3*x^2 + 9*x + 27)
\end{lstlisting}

\begin{center}
Podstawianie wartości do zmiennej -- \textbf{subs}\\
\end{center}


\begin{lstlisting}
>> myexp = x^3 + 3*x^2 - 2

>> subs(myexp,3)
\end{lstlisting}

Jeśli w wyrażeniu jest wiele zmiennych, zostanie ona wybrana domyślnie.
Można też określić, do któ©ej zmiennej chcemy podstawiać.
\begin{lstlisting}
>> syms a b x
>> varexp = a*x^2 + b*x;
>> subs(varexp,3)
ans =
9*a+3*b
>> subs(varexp,'a',3)
\end{lstlisting}


\begin{center}
Ułamki\\
\end{center}

\textbf{sym} zachowuje pierwotną postać ułamka:

\begin{lstlisting}
>> 1/3 + 1/2

>> sym(1/3 + 1/2)

>> double(ans)
\end{lstlisting}

\textbf{numden} rozbija ułamek na licznik i mianownik

\begin{lstlisting}
>> [n, d] = numden(1/3 + 1/2)

>> [n, d] = numden((x^3 + x^2)/x)

\end{lstlisting}

\begin{center}
Wyświetlanie - \textbf{pretty}\\
\end{center}

\begin{lstlisting}
>> b = sym('x^2')

>> pretty(b)
\end{lstlisting}


\begin{center}
\textsc{ Wykresy}\\
\end{center}

\textbf{ezplot} rysuje wykres z dziedziną $[-2 \pi, 2 \pi]$ funkcji podanej w nawiasie.

\begin{lstlisting}
>> ezplot('x^3 + 3*x^2 - 2')
\end{lstlisting}

Ale dziedzinę można też określić samemu:

\begin{lstlisting}
>> ezplot('cos(x)',[0 pi])
\end{lstlisting}


\begin{center}
\textsc{ Funkcja \textbf{solve}}\\
\end{center}

Podajemy równanie, które chcemy rozwiązać.

\begin{lstlisting}
>> solve('3*x^2 + x')
\end{lstlisting}

Gdy jest wiele zmiennych, Matlab sam wybiera, względem której będzie rozwiązywać.
Priorytetem jest $x$.

\begin{lstlisting}
>>solve('a*x^2 + b*x')
\end{lstlisting}

Możemy też sami podać niewiadomą.

\begin{lstlisting}
>> solve('a*x^2 + b*x','b')
\end{lstlisting}

Tyle samo równań, co niewiadomych:

\begin{lstlisting}
>> rozwiazanie = solve('4*x-2*y+z=7','x+y+5*z=10','-2*x+3*y-z=2')
\end{lstlisting}

Aby dostać się do konkretnych składowych, dajemy kropkę:

\begin{lstlisting}
>> x = rozwiazanie.x

>> y = rozwiazanie.y

>> z = rozwiazanie.z
\end{lstlisting}

\textbf{double} konwertuje symbole do wektora liczb:

\begin{lstlisting}
>> double([x y z])
\end{lstlisting}


Zadania\\


\begin{enumerate}

\item O funkcjach w Matlabie: 
\url{https://www.mathworks.com/help/matlab/ref/function.html}


\item Napisz funkcję, która sprawdza, czy podana macierz jest kwadratowa i zwraca \textbf{true} lub \textbf{false}.

\item Napisz funkcję, która przyjmuje liczbę $n$ i zwraca górnotrójkątną macierz losowych liczb całkowitych wymiaru $n$.

\item W pewnym obwodzie mamy napięcia $V_1, V_2, V_3$.
Utwórz z poniższych równań macierz i rozwiąż je, używając: $x = A^{-1}b$.

$$V_1 = 5\\
-6V_1 + 10V_2 -3V_3 = 0\\
-V 2 +51V_3 = 0$$



\item Dla poniższego układu wyrysuj proste i znajdź ich przecięcie -- rozwiązanie układu.

$$-3x_1 +x_2 = 2
-6x_1 + 2x_2 = 4$$

Użyj \textbf{solve}, by znaleźć rozwiązanie.\\
Znajdź wyznacznik macierzy układu.

Ile jest rozwiązań?

\item Zapisz układ w symbolicznej postaci i rozwiąć go funkcją \textbf{solve}.

$$2x_1 + 2x_2 + x_3 = 2\\
x_2 + 2x_3 = 1\\
x_1 + x_2 + 3x_3 = 3$$

Mając rozwiązanie w formie symbolicznej, utwórz wektor liczbowy, używając \textbf{double}.


\item Przeczytaj:

\url{https://www.mathworks.com/help/symbolic/solve-a-single-differential-equation.html?requestedDomain=www.mathworks.com}



\end{enumerate}
\begin{bottompar}
\begin{flushright}
Paulina Pełszyńska
\end{flushright}
\end{bottompar}
\end{document}
