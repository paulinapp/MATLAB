\documentclass[12pt]{amsbook}

\usepackage{amssymb} 
\usepackage{graphicx} 
\usepackage{minted}
\usepackage{url}
\usepackage{xcolor}
\definecolor{bookColor}{RGB}{19,9,46}  % 0.90\% of black
\color{bookColor}

\usepackage{lipsum}
\newenvironment{bottompar}{\par\vspace*{\fill}}{\clearpage}


% tekst pisany jest pochyłą czcionkś:
\newtheorem{twierdzenie}{Twierdzenie}[chapter] % to jest główne środowisko, ono musi być zdefiniowane jako pierwsze
\newtheorem{lemat}[twierdzenie]{Lemat} 
% tekst pisany jest prostą czcionką:
\theoremstyle{definition}
\newtheorem{definicja}[twierdzenie]{Definicja}
\newtheorem{obserwacja}[twierdzenie]{Obserwacja}
\newtheorem{stw}[twierdzenie]{Stwierdzenie}
\newtheorem{uwaga}[twierdzenie]{Uwaga}
\newtheorem{przyklad}[twierdzenie]{Przyk\l{}ad}

\usepackage{mathrsfs}
\usepackage[T1]{fontenc}
\usepackage[polish]{babel}
\usepackage[utf8]{inputenc}
\usepackage{lmodern}
\selectlanguage{polish}
\usepackage{listings}
\usepackage{matlab-prettifier}
\lstset{style=Matlab-editor}

\usepackage{scalerel}[2016/12/29]

\newcommand\Searrow{\scaleobj{0.7}{\searrow}}

\setlength\parindent{20pt}

\renewcommand\citeform[1]{{#1}}


\usepackage{geometry}
\geometry{margin=1.25in}
\usepackage{fancyhdr}

\usepackage{xcolor}
\newcommand\textlcsc[1]{\textsc{\MakeLowercase{#1}}}


\begin{document}

\begin{center}
\large{Laboratorium 6}\\
\today
\end{center}

\hspace{3cm}

\begin{center}
\textsc{Analiza Danych w Matlabie}\\
\end{center}

\hspace{3cm}


\begin{center}
Titanic\\
\end{center}

\begin{enumerate}
\item Ze strony:

\url{https://www.kaggle.com/c/titanic/data}

Pobieramy pliki: train.csv oraz test.csv

\item Importujemy dane i patrzymy, jak wyglądają:\\

\begin{lstlisting}
Train = readtable('train.csv','Format','%f%f%f%q%C%f%f%f%q%f%q%C');
Test = readtable('test.csv','Format','%f%f%q%C%f%f%f%q%f%q%C');
disp(Train(1:5,[2:3 5:8 10:11]))
\end{lstlisting}

\item Sprawdźmy, jaki wpływ na przeżycie miała płeć:\\

\begin{lstlisting}
disp(grpstats(Train(:,{'Survived','Sex'}), 'Sex'))

\end{lstlisting}

Jeśli w naszym modelu przewidujemy, że przeżyły kobiety, a mężczyźni nie, dokładność tego modelu wyniesie $78,68\%$.
Dostaniemy jeszcze niższy wynik na test.csv.

\begin{lstlisting}
gendermdl = grpstats(Train(:,{'Survived','Sex'}), {'Survived','Sex'})
all_female = (gendermdl.GroupCount('0_male') + gendermdl.GroupCount('1_female'))...
    / sum(gendermdl.GroupCount)

\end{lstlisting}

\item Missing values\\

Zapewne zauważyliście, że w kolumnie Cabin brakuje niektórych wierszy.

Zobaczmy, czy są jakieś inne zmienne z brakującymi danymi.

Przy okazji zajmiemy się wartościami dziwnymi (jak np. zerowa opłata za rejs).

\begin{lstlisting}
Train.Fare(Train.Fare == 0) = NaN;      % treat 0 fare as NaN
Test.Fare(Test.Fare == 0) = NaN;        % treat 0 fare as NaN
vars = Train.Properties.VariableNames;  % extract column names

figure
imagesc(ismissing(Train))
ax = gca;
ax.XTick = 1:12;
ax.XTickLabel = vars;
ax.XTickLabelRotation = 90;
title('Missing Values')
\end{lstlisting}


Mamy zatem $177$ pasażerów z nieznanym wiekiem.\\
Jest parę sposobów na radzenie sobie z brakującymi wartościami. Czasem po prostu usuwa się całe wiersze.
Tutaj jednak zastąpimy je średnią:

\begin{lstlisting}

avgAge = nanmean(Train.Age)             % get average age
Train.Age(isnan(Train.Age)) = avgAge;   % replace NaN with the average
Test.Age(isnan(Test.Age)) = avgAge;     % replace NaN with the average

\end{lstlisting}

Mamy 15 pasażerów z nieznaną opłatą za przejazd.
Rozsądne jest założenie, że opłaty różniły się w zależności od klasy.


\begin{lstlisting}

fare = grpstats(Train(:,{'Pclass','Fare'}),'Pclass');   % get class average
disp(fare)
for i = 1:height(fare) % for each |Pclass|
    % apply the class average to missing values
    Train.Fare(Train.Pclass == i & isnan(Train.Fare)) = fare.mean_Fare(i);
    Test.Fare(Test.Pclass == i & isnan(Test.Fare)) = fare.mean_Fare(i);
end
\end{lstlisting}

Jeśli idzie o numer Cabin, zauważmy, że tylko pasażerowie pierwszej klasy mieli kilka pokoi.
Brakujące wartości będziemy tu więc traktowali jako $0$. Podobnie z klasą trzecią, tam numery kabin są nieregularne.

\begin{lstlisting}

% tokenize the text string by white space
train_cabins = cellfun(@strsplit, Train.Cabin, 'UniformOutput', false);
test_cabins = cellfun(@strsplit, Test.Cabin, 'UniformOutput', false);

% count the number of tokens
Train.nCabins = cellfun(@length, train_cabins);
Test.nCabins = cellfun(@length, test_cabins);

% deal with exceptions - only the first class people had multiple cabins
Train.nCabins(Train.Pclass ~= 1 & Train.nCabins > 1,:) = 1;
Test.nCabins(Test.Pclass ~= 1 & Test.nCabins > 1,:) = 1;

% if |Cabin| is empty, then |nCabins| should be 0
Train.nCabins(cellfun(@isempty, Train.Cabin)) = 0;
Test.nCabins(cellfun(@isempty, Test.Cabin)) = 0;
\end{lstlisting}

Dla dwóch pasażerów nie wiemy, skąd wypłynęli. Użyjemy najczęście występującego miejsca.
Zmienimy to też na wartość liczbową, by łatwiej z tego później korzystać:

\begin{lstlisting}
% get most frequent value
freqVal = mode(Train.Embarked);

% apply it to missling value
Train.Embarked(isundefined(Train.Embarked)) = freqVal;
Test.Embarked(isundefined(Test.Embarked)) = freqVal;

% convert the data type from categorical to double
Train.Embarked = double(Train.Embarked);
Test.Embarked = double(Test.Embarked);
\end{lstlisting}


Podobnie zrobimy z płcią:

\begin{lstlisting}

Train.Sex = double(Train.Sex);
Test.Sex = double(Test.Sex);

\end{lstlisting}

Usuńmy zmienne, którch używać nie będziemy, bo zawierają dane unikatowe lub brakujące:

\begin{lstlisting}

Train(:,{'Name','Ticket','Cabin'}) = [];
Test(:,{'Name','Ticket','Cabin'}) = [];

\end{lstlisting}

\item Eksploracja danych i wzualizacja\\

Przeanalizujmy rozkład danych.

Jest to zadanie bardzo czasochłonne, ale i ważne.\\
Tu zawęzimy się do zmiennej wieku:

\begin{lstlisting}

figure
histogram(Train.Age(Train.Survived == 0))   % age histogram of non-survivers
hold on
histogram(Train.Age(Train.Survived == 1))   % age histogram of survivers
hold off
legend('Didn''t Survive', 'Survived')
title('The Titanic Passenger Age Distribution')
\end{lstlisting}


\item Cechy -- wykorzystujemy powyższe wizualizacje

Dzielimy dane na grupy wzgl. danej zmiennej, czyli u nas np. wieku.

\begin{lstlisting}

% group values into separate bins
Train.AgeGroup = double(discretize(Train.Age, [0:10:20 65 80], ...
    'categorical',{'child','teen','adult','senior'}));
Test.AgeGroup = double(discretize(Test.Age, [0:10:20 65 80], ...
    'categorical',{'child','teen','adult','senior'}));

\end{lstlisting}

Popatrzmy teraz na opłatę za bilet.

\begin{lstlisting}

figure
histogram(Train.Fare(Train.Survived == 0));         % fare histogram of non-survivers
hold on
histogram(Train.Fare(Train.Survived == 1),0:10:520) % fare histogram of survivers
hold off
legend('Didn''t Survive', 'Survived')
title('The Titanic Passenger Fare Distribution')

% group values into separate bins
Train.FareRange = double(discretize(Train.Fare, [0:10:30, 100, 520], ...
    'categorical',{'<10','10-20','20-30','30-100','>100'}));
Test.FareRange = double(discretize(Test.Fare, [0:10:30, 100, 520], ...
    'categorical',{'<10','10-20','20-30','30-100','>100'})); 

\end{lstlisting}


\end{enumerate}







źródło: \url{https://blogs.mathworks.com/loren/2015/06/18/getting-started-with-kaggle-data-science-competitions/}


\begin{bottompar}
\begin{flushright}
Paulina Pełszyńska
\end{flushright}
\end{bottompar}
\end{document}
