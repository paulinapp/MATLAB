\documentclass[12pt]{amsbook}

\usepackage{amssymb} 
\usepackage{graphicx} 
\usepackage{minted}
\usepackage{url}
\usepackage{xcolor}
\definecolor{bookColor}{RGB}{19,9,46}  % 0.90\% of black
\color{bookColor}

\usepackage{lipsum}
\newenvironment{bottompar}{\par\vspace*{\fill}}{\clearpage}


% tekst pisany jest pochyłą czcionkś:
\newtheorem{twierdzenie}{Twierdzenie}[chapter] % to jest główne środowisko, ono musi być zdefiniowane jako pierwsze
\newtheorem{lemat}[twierdzenie]{Lemat} 
% tekst pisany jest prostą czcionką:
\theoremstyle{definition}
\newtheorem{definicja}[twierdzenie]{Definicja}
\newtheorem{obserwacja}[twierdzenie]{Obserwacja}
\newtheorem{stw}[twierdzenie]{Stwierdzenie}
\newtheorem{uwaga}[twierdzenie]{Uwaga}
\newtheorem{przyklad}[twierdzenie]{Przyk\l{}ad}

\usepackage{mathrsfs}
\usepackage[T1]{fontenc}
\usepackage[polish]{babel}
\usepackage[utf8]{inputenc}
\usepackage{lmodern}
\selectlanguage{polish}
\usepackage{listings}
\usepackage{matlab-prettifier}
\lstset{style=Matlab-editor}

\usepackage{scalerel}[2016/12/29]

\newcommand\Searrow{\scaleobj{0.7}{\searrow}}

\setlength\parindent{20pt}

\renewcommand\citeform[1]{{#1}}


\usepackage{geometry}
\geometry{margin=1.25in}
\usepackage{fancyhdr}

\usepackage{xcolor}
\newcommand\textlcsc[1]{\textsc{\MakeLowercase{#1}}}


\begin{document}

\begin{center}
\large{Laboratorium 5}\\
\today
\end{center}

\hspace{3cm}

\begin{center}
\textsc{Jeszcze więcej o wykresach}\\
\end{center}

\hspace{3cm}


\begin{center}
2--wymiarowe\\
\end{center}

\url{https://www.mathworks.com/help/symbolic/fplot.html}



\hspace{3cm}


\begin{center}
3--wymiarowe\\
\end{center}

\url{https://www.mathworks.com/help/symbolic/fsurf.html}



\hspace{3cm}

\begin{center}
\textsc{Zadania}\\
\end{center}

\begin{enumerate}

\item Proszę wykreślić w trójwymiarze funkcję $\cos(x) \cdot \sin(y)$ \\
dla $x \in [-3,3]$ i $y \in [-4,4]$.
 Do stworzenia wykresu można użyć dowolnej funkcji pozwalającej na kreślenie
wykresów 3d.

\item Proszę narysować symbolicznie trójwymiarowy wykres funkcji Rastrigina:

$$ y = x_1^2 + x_2^2 - \cos(12x_1) - \cos (18x_2), \ \ x_1 \in [-1,1], \ \ x_2 \in [-1,1]$$

\end{enumerate}


\begin{bottompar}
\begin{flushright}
Paulina Pełszyńska
\end{flushright}
\end{bottompar}
\end{document}
